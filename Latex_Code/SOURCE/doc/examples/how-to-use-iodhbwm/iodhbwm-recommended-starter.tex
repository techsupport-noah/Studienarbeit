% ---------------------------------------------------
% Date:       15.05.2019
% Version:    v1.1.0
% Autor:      Felix Faltin <ffaltin91[at]gmail.com>
% Repository: https://github.com/faltfe/iodhbwm
% ---------------------------------------------------
% --- --- --- --- -- Class options -- --- --- --- ---
% ---------------------------------------------------
\documentclass[
    load-dhbw-templates,                  % Allow \dhbw* commands
    auto-intro-pages,                     % Takes care about titlepage, abstract, ToC, etc.
    add-tocs-to-toc,                      % Add LoF, LoT, etc. to ToC
    add-bibliography,                     % Include bibliography (needs biber run)
    bib-file     = biblatex-examples.bib, % Set bibliography file
    language     = english,               % Set english as second language
    mainlanguage = ngerman,               % Set main document language
    debug                                 % Provide \lipsum, \blindtext
]{iodhbwm}
\usepackage[T1]{fontenc}

% ---------------------------------------------------
% --- --- --- --- - Necessary setup - --- --- --- ---
% ---------------------------------------------------
\dhbwsetup{%
    author            = Max Mustermann,
    thesis type       = PA,
    thesis title      = Einführungsbeispiel mit empfohlenen Einstellungen,
    student id        = 12345,
    location          = Musterstadt,
    institute         = Musterwerke GmbH,
    course/id         = Txxxx,
    supervisor        = Felix Faltin,
    processing period = {01.01.17 -- 31.01.17},
}

\begin{document}
    % ---------------------------------------------------
    % --- --- --- --- Begin actual content -- --- --- ---
    % ---------------------------------------------------
    \chapter{Test Kapitel}\label{chap:test-chap}
        \section{Test Section}\label{sec:test-sec}
            \blindtext
    
            Verweis auf \Cref{chap:test-chap} und auf mich selbst \Cref{sec:test-sec}~\cite{glashow}.
    
            \lipsum
    % ---------------------------------------------------
    % --- --- --- --- End actual content --- --- --- ---
    % ---------------------------------------------------
    % There is no need to include the bibliography
    % manually. The class takes care about the correct
    % position. You only have to run
    %   pdfLaTeX -> biber -> 2x pdfLaTeX
    % ---------------------------------------------------
\end{document}
